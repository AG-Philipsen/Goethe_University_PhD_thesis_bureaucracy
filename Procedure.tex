\documentclass[a4paper]{article}
\usepackage{xcolor}
\usepackage{eurosym}
\usepackage[textwidth=16cm,textheight=26cm]{geometry}
\usepackage[colorlinks,filecolor=violet,urlcolor=blue,%
            pdftitle={PhD thesis bureaucracy}]{hyperref}
\usepackage{multicol}


\newlength{\titleVerticalSpacing}
\newcommand{\myTitle}[2]{%
    \begingroup
        \titleVerticalSpacing = 0.03\textheight
        \centering
        \vspace*{\titleVerticalSpacing}
        {\Huge\bfseries #1}\\[\baselineskip]
        {\itshape #2}\\[3\baselineskip]
    \endgroup
}


\begin{document}
    
    \myTitle{PhD thesis bureaucracy}{January 2017}

    \section*{Requirements for the thesis}
        There are not that many requirements when it comes to style, just few that must be fulfilled:
        \begin{itemize}
            \item The \textcolor{red}{first $2$ pages} are fixed, I am sure you have seen these in every other thesis.
                  In \LaTeX, these can be produced using the files \texttt{Titlepage.tex} and \texttt{Titlepage\_back.tex} (after having filled them out).
                  Of course, you can make them yourself even better.
            \item You need a \textcolor{red}{$5$ page long} German summary. Note that at the Promotionsb\"uro they are strict on the number of pages, but it has not been a problem so far if the fifth page is not completely full.
            \item You need \textcolor{red}{a CV} at the end of your thesis which has to include your teaching experiences and a list of publications (it does not need a photograph).
        \end{itemize}
        
        
    \section*{Printing the Thesis}
        \begin{itemize}
            \item A possibility to print your thesis is to use the \href{http://druck.uni-frankfurt.de/druckzentrum/}{university print service}.
                  Some of the past students were very pleased with the result.
                  In general it takes a couple of days.
            \item A more professional, though more expensive alternative is to use a professional private service.
                  There are many in the Frankfurt area, so feel free to google them.
                  One extraordinarily professional shop\footnote{You could need it if you have very particular figures (e.g. with fancy shadings) in your Thesis.} is in Darmstadt and it has a very nice \href{http://www.der-felger.de/}{website}.
                  You can directly order online and receive your Thesis at home (but you can also pick it up yourself).
                  They offer quite a lot of options for the binding and they are very fast (if you make your order in the morning, you can pick it up in the afternoon). Furthermore, every first and fifteenth day of the month, they offer a 20\% discount on the printing.
        \end{itemize}
        
    
    \section*{Summary of needed documents for the submission}
        Most of the information can be gathered from the \href{https://www.uni-frankfurt.de/42800906/startseite}{\texttt{Promotionsb\"uro's website}}, more specifically their \href{https://www.uni-frankfurt.de/42800991/downloads}{\texttt{download page}}.
        So what you need to hand in to the Promotionsb\"uro is:
        \begin{itemize}
            \item $7$ copies of your dissertation ($5 + 2$), where they will stamp two of them for you to take back to your sensors;
            \item $1$ burnt CD with the dissertation on it (you can produce a nice cover using the \href{run:Cover.tex}{file} in the \texttt{CD\_cover} directory in the repository);
            \item A separate $2$ page German summary, which is in addition to the $5$ page summary in your thesis;
            \item A \href{run:Erlaerung_Formular.pdf}{signed declaration} that you have carried out the work independently;
            \item A \href{run:Antrag_Formular.pdf}{signed request} to be admitted for the assessment process;
            \item A CV with a photograph.
        \end{itemize}
        So basically you just walk over to the Promotionsb\"uro (see \texttt{Promotionsbuero\_Anfahrtsplan.pdf} file in the repository) and hand it all in.
        They will give you a bill of \euro$\,150$ which you have to pay, and that is basically that.
        At a later point you need to fix your defence and find a committee.

    \section*{Organising a defence}
      After submitting your thesis for evaluation, your sensors will be given a
      one month period to assess the thesis\footnote{Although it can take
        longer, the Promotionsb\"uro doesn't want to be rude to the
        professors.}. After this period, the 4-6 week thesis circulation can
      begin. At this stage, you should fix a committee as well as a date
      for your defence. You can either find a committee first, and then try to
      all agree on a date, or fix a date (or a set of possible dates) with your
      sensors, and then find a committee to fit your available dates.

      The committee must consist of four faculty employees, three of which must
      be professors. Two of them will be your sensors, while \textcolor{red}{the
        two others must be from institutes different from your own}. The six
      institutes of the faculty of physics are%
      \footnote{Can someone verify that you can choose a committee from all of
        these?}
      %
      \begin{multicols}{2}
        \begin{itemize}
          \item Institut f\"ur Angewandte Physik (IAP),
          \item Institut f\"ur Biophysik (IFB),
          \item Institut f\"ur Didaktik der Physik (IDP),
          \item Institut f\"ur Kernphysik (IKF),
          \item Institut f\"ur Theoretische Physik (ITP),
          \item Physikalisches Institut (PI).
        \end{itemize}
      \end{multicols}
      %
      On top of the four members of your committee, you also need to find
      replacements for the two additional assessors, should they fail to show up
      for the defence. So all in all you need \textcolor{red}{six members of
        faculty staff, two will be your sensors, two will be staff from a
        different institute, two will be their replacements}. When the committee
      has been organised, the time and date has been fixed, and a room has been
      booked, you need to fill in the \href{run:Prufungskommission.pdf}{defence
        committee form}. As one can see from the form, you need to appoint a
      chairman, as well as someone to write the PhD protocol. This form must
      then be approved and signed by the deanery, who will then forward it to
      the promotionsb\"uro. This must happen before the circulation can
      commence, and must therefore be in order at least four weeks before the
      date of the defence%
      \footnote{I cannot find any mention of this in the PhD regulation document
        on the promotionsb\"uro homepage, but I was yelled at for handing this
        in a week before my defence. - \emph{Jonas}}.
    
\end{document}
