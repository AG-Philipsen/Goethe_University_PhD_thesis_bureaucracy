\documentclass[a4paper]{article}
\usepackage[svgnames]{xcolor}
\usepackage{eurosym}
\usepackage{multicol}
\usepackage{siunitx}
\usepackage{xifthen}
\usepackage{datetime}
\usepackage[textwidth=15cm,textheight=23cm]{geometry}
\usepackage[pdftitle={PhD thesis bureaucracy}]{hyperref}

%Colors for hyperref setup, but also to be used later in consinstent way
\colorlet{fileColor}{violet}
\colorlet{urlColor}{blue}
\colorlet{linkColor}{blue!50!black}
\hypersetup{colorlinks, filecolor=fileColor, urlcolor=urlColor, linkcolor=linkColor}

%Command for heading
\newlength{\titleVerticalSpacing}
\newcommand{\myTitle}[2]{%
    \begingroup
        \titleVerticalSpacing = 0.03\textheight
        \centering
        \vspace*{\titleVerticalSpacing}
        {\Huge\bfseries #1}\\[\baselineskip]
        {\itshape #2}\\[3\baselineskip]
    \endgroup
}
\newdateformat{monthyeardate}{%
  \monthname[\THEMONTH] \THEYEAR}

%Commands for links, files, etc.
\newcommand{\attention}[1]{\textcolor{red}{#1}}
\newcommand{\file}[2][]{%
    \ifthenelse{\isempty{#1}}%
        {\textcolor{fileColor}{#2}}%
        {\href{run:#1}{\texttt{\textcolor{fileColor}{#2}}}}%
}
\newcommand{\website}[2][]{%
    \ifthenelse{\isempty{#1}}%
        {\textcolor{urlColor}{#2}}%
        {\href{#1}{\texttt{\textcolor{urlColor}{#2}}}}%
}

\begin{document}
    
    \myTitle{PhD thesis bureaucracy}{\monthyeardate\today}

    \section*{Requirements for the thesis}
        There are not that many requirements when it comes to style, just few that must be fulfilled:
        \begin{itemize}
            \item The \attention{first $2$ pages} are fixed, I am sure you have seen these in every other theses.
                  In \LaTeX, these can be produced using the files \file{Titlepage.tex} and \file{Titlepage\_back.tex} (after having filled them out).
                  You can alternatively make them yourself. The requirement is on content, not on style and spacing.
            \item You need a \attention{$\geq 5$ page long} German summary. This is a hard limit, but our experiences show that the fifth page need
              only be partially filled.
            \item You need \attention{a CV} at the end of your thesis which has to include your teaching experiences and a list of publications
              (it does not require a photograph).
        \end{itemize}
        
        
    \section*{Printing the thesis}
        You can print your thesis in whatever way you want. Here are a few alternatives:
        \begin{itemize}
            \item You can use the \website[https://www.rz.uni-frankfurt.de/45046455/Druckzentrum/]{university print service}.
                  Some of the previous students have been very pleased with the result. In general it takes a couple of days,
                  and costs approximately between \euro$\,15$ and \euro$\,20$ per copy.
            \item A more expensive alternative is to use a professional private service.
                  There are many in the Frankfurt area, so feel free to google them.
                  One extraordinarily professional shop\footnote{You could need it if you have very particular figures (e.g.
                    with fancy shadings) in your thesis.} is in Darmstadt and it has a very nice \website[http://www.der-felger.de/]{website}.
                  You can directly order online and receive your thesis at home (but you can also pick it up yourself).
                  They offer quite a lot of options for the binding and they are very fast (if you make your order in the morning, you can
                  pick it up in the afternoon). Furthermore, every first and fifteenth day of the month, they offer a $20$\% discount on the printing.
            \item A third (and cheapest) alternative is to buy the required \SI{100}{\gram} paper yourself and print your thesis using the university
              printers. The printed pages can then be bound with a standard glue binding machine, or something equivalent. This should be done during
              low activity hours, as it will most likely take a substantial amount of time.
        \end{itemize}
        
    
    \section*{Summary of needed documents for the submission}
        Most of the information can be gathered from the \website[https://www.uni-frankfurt.de/42800906/startseite]{Promotionsb\"uro's website}, more specifically their \website[https://www.uni-frankfurt.de/42800991/downloads]{download page}.
        Below is a summary of what you need to hand off to the Promotionsb\"uro:
        \begin{itemize}
            \item $7$ copies of your dissertation ($5 + 2$), where they will stamp two of them for you to take back to your sensors;
            \item $1$ burnt CD with the dissertation on it (you can produce a nice cover using the \file{file} in the \texttt{CD\_cover} directory in the repository);
            \item A separate $2$ page German summary, which is in addition to the $5$ page summary in your thesis;
            \item A \file[Erklaerung_Formular.pdf]{signed declaration} that you have carried out the work independently;
            \item A \file[Antrag_Formular.pdf]{signed request} to be admitted for the assessment process;
            \item A CV with a photograph.
            \item A certified copy of the Master diploma (or just a copy of it to be presented with the original one).
                  In case you did not obtain your Master degree in Germany, you need to bring there a certified
                  translation of your diploma which must contain your Master degree mark (either to English or to
                  German)\footnote{They were not so strict with me about this point and, indeed, they had just a quick
                  look to my diploma. - \emph{Alessandro}}.
        \end{itemize}
        So basically you just \file[Promotionsbuero\_Anfahrtsplan.pdf]{walk over} to the Promotionsb\"uro and hand it all in.
        They will give you a bill of \euro$\,150$ which you have to pay, and that is basically that.
        At a later point you need to fix your defence and find a committee.

    \section*{Organising a defence}
        After submitting your thesis for evaluation, your sensors will be given a one month period to assess the thesis\footnote{Although it can take longer, the Promotionsb\"uro does not want to be rude to the professors.}.
        After this period, the $4$-$6$ week thesis circulation can begin.
        At this stage, you should fix a committee as well as a date for your defence.\footnote{There exists also a corresponding \file[Merkblatt_Physik.pdf]{information sheet} (in German only), which can be found on the download page of the Promotionsb\"uro.}
        You can either find a committee first, and then try to all agree on a date, or fix a date (or a set of possible dates) with your sensors, and then find a committee to fit your available dates.

        The committee must consist of four faculty employees, three of which must be professors.
        Two of them will be your sensors, while \attention{the two others must be from institutes different from your own}.
        The six institutes of the faculty of physics are\footnote{Can someone verify that you can choose a committee from all of these?}
        \begin{multicols}{2}
          \begin{itemize}
            \item Institut f\"ur Angewandte Physik (IAP),
            \item Institut f\"ur Biophysik (IFB),
            \item Institut f\"ur Didaktik der Physik (IDP),
            \item Institut f\"ur Kernphysik (IKF),
            \item Institut f\"ur Theoretische Physik (ITP),
            \item Physikalisches Institut (PI).
          \end{itemize}
        \end{multicols}
        \noindent On top of the four members of your committee, you also need to find replacements for the two additional assessors, should they fail to show up for the defence.
        So all in all you need \attention{six members of faculty staff, two will be your sensors, two will be staff from a different institute, two will be their replacements}.
        When the committee has been organised, the time and date has been fixed, and a room has been booked, you need to fill in the \file[Pruefungskommission.pdf]{defence committee form}.
        As one can see from the form, you need to appoint a chairman, as well as someone to write the PhD protocol.
        This form must then be approved and signed by the deanery, who will then forward it to the promotionsb\"uro.
        This must happen before the circulation can commence, and must therefore be in order at least four weeks before the date of the defence\footnote{I cannot find any mention of this in the PhD regulation document on the promotionsb\"uro homepage, but I was yelled at for handing this in a week before my defence. - \emph{Jonas}}.
        
        Consider to speak with the members of your committee about how they would like to read your thesis.
        Normally sending the \texttt{pdf} file around should be enough. 
        According to the \website[http://www.uni-frankfurt.de/42801290/mitteilungen]{important notes} of the Promotionsb\"uro, you are supposed to give one copy \emph{only} to each of your referees.
        Nevertheless it seems that also the other members of the committee should get one, even though it is not stated anywhere than you should be the person providing them with a copy\footnote{I was reproached with not having given a copy of my thesis to one member of the commission, which asked the promotionsb\"uro about how to get one. - \emph{Alessandro}}.


    \section*{Getting your certificate}
         In Germany, until you do not get your certificate, you cannot say that you are a Doctor.
         Only other people who knows you successfully defended can.
         Fortunately, it is not so hard to get it.
         Basically you have to hand in your thesis to the library, get a confirmation about your delivery and bring this confirmation to the Promotionsb\"uro where you will get your certificate.
         Some further information could be useful.
         \begin{itemize}
            \item According to the \website[http://www.ub.uni-frankfurt.de/dissertationen/abgabe.html\#online]{library website}, you have several ways to hand in your thesis.
                  Feel free to choose any of them.
                  The least obsolete way to do so is to
                  \begin{itemize}
                      \item \website[http://publikationen.ub.uni-frankfurt.de/publish]{upload online} the pdf file of your thesis,
                      \item fill out, print and sign the \file[Einverstaendniserklaerung.pdf]{authorisation to publish} your thesis,
                      \item burn 1 CD with the dissertation on it as done for the submission,
                      \item prepare two printed and bound copy of your thesis
                  \end{itemize}
                  and bring everything \website[https://www.ub.uni-frankfurt.de/zentrale/so_en.html]{to the library}.
                  It is important to note that \textbf{no staple or spiral binding} will be accepted.
            \item It is possible to get back from the Promotionsb\"uro the printed copies you handed in before your defense (you will get fewer back in case some Professor decided to keep one).
            \item Together with the original certificate in German, you will get two duplicates and one English translation of it.
        \end{itemize}

        \bigskip
        \begin{center}
            \fbox{\parbox[c][3em][c]{0.9\textwidth}{\centering\large Congratulations! You are a doctor and you can say it! \texttt{\attention{Ph}}inally \texttt{\attention{D}}one!}}
        \end{center}

\end{document}
